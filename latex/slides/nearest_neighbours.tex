\section{Nearest Neighbours}
\subsection{Idea}
\begin{frame}
    \frametitle{Nearest Neighbours}
    \textbf{putting it all together\dots}
    \begin{itemize}
        \item for each item: calculate morton key \\
        \item for a given query point: calculate neighbour candidates on morton
            curve \\
        \item for each candidate: \\
        \begin{itemize}
            \item is the candidate on a boundary? If yes, continue with next \\
            \item search the candidate's node in the tree $\implies$ 3 cases \\
            \begin{enumerate}
                \item candidate exists and is end node \textrightarrow take as
                    neighbour and continue \\
                \item candidate doesn't exist \textrightarrow take last
                    existing parent node as neighbour and continue \\
                \item candidate exists and has further children \textrightarrow
                    search its children recursively, facing in the direction of
                    the query point
            \end{enumerate}
        \end{itemize}
    \end{itemize}
\end{frame}

\subsection{Examples}
\begin{frame}
    \textbf{Example:} Query point = (6, 5)
    \begin{figure}
        \centering
        \begin{animateinline}[controls={step,play,stop},buttonsize=10pt]{1}
            \multiframe{4}{i=1+1}{%
                \resizebox{.7\textwidth}{!}{%
                    \begin{tikzpicture}
\drawpoints
\drawquadtree
\fill [red] (B) circle (3pt);
\foreach \point in {A, C, D, E, F, G, H}
    \fill [black] (\point) circle (3pt);

\ifthenelse{4 > \i > 1 }{%
    \begin{pgfonlayer}{background}
        \fill[orange!20] (6,4) -- (6,6) -- (8,6) -- (8,4) -- cycle;
    \end{pgfonlayer}
}{}

\ifthenelse{\i = 2}{%
    \begin{pgfonlayer}{background}
        \fill[blue!20] (6,2) -- (8,2) -- (8,4) -- (6,4) -- cycle;
        \fill[blue!20] (8,4) -- (10,4) -- (10,6) -- (8,6) -- cycle;
        \fill[blue!20] (8,6) -- (8,8) -- (6,8) -- (6,6) -- cycle;
        \fill[blue!20] (6,4) -- (4,4) -- (4,6) -- (6,6) -- cycle;
    \end{pgfonlayer}
}{}

\ifthenelse{\i = 3}{%
    \begin{pgfonlayer}{background}
        \fill[blue!20] (8,6) -- (8,8) -- (6,8) -- (6,6) -- cycle;
        \fill[blue!20] (4,0) -- (8,0) -- (8,4) -- (4,4) -- cycle;
        \fill[blue!20] (8,0) -- (16,0) -- (16,8) -- (8,8) -- cycle;
    \end{pgfonlayer}
}{}

\ifthenelse{\i = 4}{%
    \coordinate [label=315:\nodelabel{(6,5)}] (pointB) at (6.5, 5.5);
    \coordinate [label=right:\nodelabel{(5,2)}] (pointA) at (5.5, 2.5);
    \coordinate [label=225:\nodelabel{(7,7)}] (pointD) at (7.5, 7.5);
    \coordinate [label=right:\nodelabel{(9,6)}] (pointE) at (9.5, 6.5);
    \fill[blue] (A) circle (3pt);
    \fill[blue] (D) circle (3pt);
    \fill[blue] (E) circle (3pt);
}{}
\end{tikzpicture}

                }
            }
        \end{animateinline}
    \end{figure}
\end{frame}

\begin{frame}
    \begin{figure}
        \centering
        \begin{tikzpicture}[tree layout]
        \graph {%
            ROOT [rn] -- {[fresh nodes]
                / [branch] -- {A [branch] -- {/ [candidate]}, / [branch] -- {%
                        / [candidate], B [query], C [branch], D [candidate]
                    }
                },
                E [branch] -- {/ [branch] -- {/ [candidate]}}, / [branch] -- {%
                    F [branch], G [branch], H [branch]
                }
            }
        };
        \end{tikzpicture}
        \caption{candidate nodes}
    \end{figure}
    \begin{figure}
        \centering
        \begin{tikzpicture} [tree layout]
        \graph {%
            ROOT [rn] -- {[fresh nodes]
                / [branch] -- {A [neighbour]}, / [branch] -- {%
                    B [query], C [branch], D [neighbour]
                },
                E [neighbour], / [branch] -- {%
                    F [branch], G [branch], H [branch]
                }
            }
        };
        \end{tikzpicture}
        \caption{actual neighbours}
    \end{figure}
\end{frame}

\begin{frame}
    \textbf{Example:} Query point = (9, 6)
    \begin{figure}
        \centering
        \begin{animateinline}[controls={step,play,stop},buttonsize=10pt]{1}
            \multiframe{4}{i=1+1}{%
                \resizebox{.7\textwidth}{!}{%
                    \begin{tikzpicture}
\drawpoints
\drawquadtree
\fill [red] (E) circle (3pt);
\foreach \point in {A, B, C, D, F, G, H}
    \fill [black] (\point) circle (3pt);

\ifthenelse{4 > \i > 1 }{%
    \begin{pgfonlayer}{background}
        \fill[orange!20] (8,0) -- (16,0) -- (16,8) -- (8,8) -- cycle;
    \end{pgfonlayer}
}{}

\ifthenelse{\i = 2}{%
    \begin{pgfonlayer}{background}
        \fill[blue!20] (0,0) -- (8,0) -- (8,8) -- (0,8) -- cycle;
        \fill[blue!20] (8,8) -- (16,8) -- (16,16) -- (8,16) -- cycle;
    \end{pgfonlayer}
}{}

\ifthenelse{\i = 3}{%
    \begin{pgfonlayer}{background}
        \fill[blue!20] (4,0) -- (8,0) -- (8,4) -- (4,4) -- cycle;
        \fill[blue!20] (6,4) -- (8,4) -- (8,8) -- (6,8) -- cycle;
        \fill[blue!20] (8,8) -- (16,8) -- (16,12) -- (8,12) -- cycle;
    \end{pgfonlayer}
}{}

\ifthenelse{\i = 4}{%
    \coordinate [label=right:\nodelabel{(9,6)}] (pointE) at (9.5, 6.5);
    \coordinate [label=right:\nodelabel{(5,2)}] (pointA) at (5.5, 2.5);
    \coordinate [label=315:\nodelabel{(6,5)}] (pointB) at (6.5, 5.5);
    \coordinate [label=225:\nodelabel{(7,7)}] (pointD) at (7.5, 7.5);
    \coordinate [label=45:\nodelabel{(8,8)}] (pointF) at (8.5, 8.5);
    \coordinate [label=right:\nodelabel{(12,9)}] (pointG) at (12.5, 9.5);
    \fill[blue] (A) circle (3pt);
    \fill[blue] (B) circle (3pt);
    \fill[blue] (D) circle (3pt);
    \fill[blue] (F) circle (3pt);
    \fill[blue] (G) circle (3pt);
}{}
\end{tikzpicture}

                }
            }
        \end{animateinline}
    \end{figure}
\end{frame}

\begin{frame}
    \begin{figure}
        \centering
        \begin{tikzpicture} [tree layout]
        \graph {%
            ROOT [rn] -- {[fresh nodes]
                / [candidate] -- {A [branch], / [branch] -- {%
                        B [branch], C [branch], D [branch]
                    }
                },
                E [query], / [candidate] -- {%
                    F [branch], G [branch], H [branch]
                }
            }
        };
        \end{tikzpicture}
        \caption{candidate nodes}
    \end{figure}
    \begin{figure}
        \centering
        \begin{tikzpicture} [tree layout]
        \graph {%
            ROOT [rn] -- {[fresh nodes]
                / [branch] -- {A [neighbour]}, / [branch] -- {%
                    B [neighbour], C [branch], D [neighbour]
                },
                E [query], / [branch] -- {%
                    F [neighbour], G [neighbour], H [branch]
                }
            }
        };
        \end{tikzpicture}
        \caption{actual neighbours}
    \end{figure}
\end{frame}

\subsection{Implementation}
\begin{frame}[fragile]
    \textbf{Finding nearest neighbours - the code}
    \begin{minted}[linenos,breaklines,tabsize=4]{c}
void find_neighbours( uint16_t key, node_t *head, dynarr_t *res )
{
    int i;
    node_t *current, *tmp;
    current = search( key, head );
    uint16_t candidates[4] = {
        left( current->key ), right( current->key ),
        top( current->key ),  bot( current->key )
    };
    \end{minted}
\end{frame}

\begin{frame}[fragile]
    \textit{cont\dots}
    \begin{minted}[linenos,firstnumber=10,breaklines,tabsize=4]{c}
    for ( i = 0; i < 4; ++i ) {
        if ( is_on_boundary( current->key, i ) )
            continue;
        tmp = search( candidates[i], head );
        if ( tmp->level == current->level && tmp->children != NULL )
            search_children( tmp, i, res );
        else if ( tmp->item != NULL )
            dynarr_append(res, tmp->item);
    }
}
    \end{minted}
\end{frame}

\begin{frame}[fragile]
    \textbf{search children recursively}
    \begin{minted}[linenos,breaklines,tabsize=4]{c}
uint16_t suffixes[4][2] = {
    { 1, 3 },   /* left */
    { 0, 2 },   /* right */
    { 2, 3 },   /* top */
    { 0, 1 }    /* bottom */
};

void search_children( node_t *head, int i, dynarr_t *res )
{
    if ( head->item != NULL) {  /* i.e. head->children == NULL */
        dynarr_append( res, head->item );
    }
    \end{minted}
\end{frame}

\begin{frame}[fragile]
    \textit{cont\dots}
    \begin{minted}[linenos,firstnumber=13,breaklines,tabsize=4]{c}
    else {
        if ( head->children[suffixes[i][0]] != NULL )
            search_children( head->children[suffixes[i][0]], i, res );
        if ( head->children[i][1] != NULL )
            search_children( head->children[suffixes[i][1]], i, res );
    }
}
\end{minted}
\end{frame}

\begin{frame}[fragile]
    \textbf{Comparing the Performance:} quadtree vs naive
    \begin{minted}[linenos,tabsize=4]{bash}
~$: bin/timeit.out

TIMEIT naive - 256 items - 100 runs
average     :   (170.9 +- 24.3) μs
    max/min :   (335.3 / 141.6) μs
total time  :   17206.8 μs

TIMEIT quadtree - 256 items - 100 runs
average     :   (154.0 +- 23.3) μs
    max/min :   (291.1 / 143.4) μs
total time  :   15534.9 μs

    \end{minted}
\end{frame}

\begin{frame}[fragile]
    \textit{cont\dots}
    \begin{minted}[linenos,firstnumber=12,tabsize=4]{bash}
~$: bin/timeit.out

TIMEIT naive - 1265 items - 100 runs
average     :   (3511.6 +- 464.3) mus
    max/min :   (7954.8 / 3382.4) μs
total time  :   351213.0 μs

TIMEIT quadtree - 1265 items - 100 runs
average     :   (891.8 +- 133.3) μs
    max/min :   (1787.1 / 833.7) μs
total time  :   89300.7 μs

    \end{minted}
\end{frame}

% vim: set ff=unix tw=79 sw=4 ts=4 et ic ai :
